\section{Measured Sky Backgrounds}
\label{sec:sky_background}

During the LSSTCam on-sky campaign, dedicated tests were performed to characterize the background, evaluate the performance of the \texttt{SkyCorrectionTask}, and verify compliance with project-level requirements.

\subsection{Sky Brightness Determination}

Sky brightness measurements were performed using forced photometry on randomly placed “sky sources’’ across the focal plane.
For each position, the local sky flux was compared to the model prediction.
Across the sample of real LSSTCam data, the precision of the sky brightness determination was better than 1\% in approximately 99\% of all measurements, meeting the system requirement (Fig.~\ref{fig:sky_backgrounds}).

These tests used data from weekly DRP processing runs (e.g., week~18) that employed the full-focal-plane background modeling introduced with the \texttt{SkyCorrectionTask}.
This task, initially developed for HSC, was commissioned for LSSTCam and now fits large-scale gradients across the entire focal plane.
Comparisons between per-detector and full-focal-plane background solutions show that the latter reduces over-subtraction near bright, extended sources and improves uniformity across the field.

\begin{figure}[h]
    \centering
    \includegraphics[width=0.4\textwidth]{figures/sky_sources.png}

    \caption{Performance of sky brightness determination using forced photometry of randomly placed sky sources.
    Nearly all measurements ($\sim$99\%) achieve better than 1\% precision, satisfying the system requirement.}
    \label{fig:sky_backgrounds}
\end{figure}

\subsection{Limiting Surface Brightness Sensitivity}

Random sampling of sky regions from ten representative $g$-band LSSTCam visits yields a limiting surface brightness of $\sim$28~mag~arcsec$^{-2}$.
This limit is defined as three times the standard deviation of the background flux distribution within 10~arcsecond boxes.
Further validation using radial photometry of the brightest cluster galaxy (BCG) in Abell~360 confirms that surface brightness features are detected to similar levels in multiple bands.
These measurements demonstrate that the system is achieving the expected depth for extended, low surface brightness structures (Fig.~\ref{fig:lsb_results}).

\begin{figure}[h]
    \centering
    \includegraphics[width=0.4\textwidth]{figures/surfacebrightness1.png}
    \includegraphics[width=0.4\textwidth]{figures/surfacebrightness2.png}
    \caption{Low surface brightness performance.
    \textit{Left:} Random sampling of $g$-band LSSTCam visits yields a limiting surface brightness of $\sim$28~mag~arcsec$^{-2}$.
    \textit{Right:} Radial photometry around the BCG in Abell~360 confirms consistent surface brightness sensitivity across filters.}
    \label{fig:lsb_results}
\end{figure}

\subsection{Optical Ghost Area Impact}

The contribution of optical ghosts to the effective background and usable imaging area was evaluated using ray-tracing simulations with the \texttt{Batoid} optical model.
Simulated stars of varying brightness were placed at different field positions, and the fractional area of the focal plane impacted by ghost reflections was measured as a function of bandpass.
The mean ghost-affected area across all filters is $\sim$0.6\%, with the largest values in $u$ and the smallest in $r$ and $i$.
This satisfies the system requirement that the fractional area impacted by ghosts remain below 1\%.
On-sky verifications confirm that the simulated ghost patterns correspond well to observed behavior (Fig.~\ref{fig:ghosts}).

\begin{figure}[h]
    \centering
    \includegraphics[width=0.5\textwidth]{figures/ghost_impact1.png}
    \includegraphics[width=0.9\textwidth]{figures/ghost_impact2.png}
    \caption{Fraction of the focal plane impacted by optical ghosts simulated with \texttt{Batoid}.
    The mean loss is $\sim$0.6\% across all filters, with worst performance in $u$ and best in $r$ and $i$, satisfying the requirement that ghost area loss remain below 1\%.}
    \label{fig:ghosts}
\end{figure}

\subsection{Summary}

The commissioning campaign demonstrated that LSSTCam background modeling meets or exceeds requirements.
Sky brightness precision is better than 1\%, limiting surface brightness sensitivity reaches $\sim$28~mag~arcsec$^{-2}$, and the fractional area affected by optical ghosts remains below 1\%.
These results validate both the photometric and low surface brightness performance of the system and establish a foundation for continued optimization of background modeling in future data releases.