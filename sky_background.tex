\section{Measured Sky Backgrounds}
\label{sec:sky_background}

During the LSSTCam on-sky campaign, dedicated tests were performed to characterize the background, evaluate the performance of the \texttt{SkyCorrectionTask}, and verify compliance with project-level requirements.

\subsection{Sky Brightness Determination}

Sky brightness measurements were performed using forced photometry on randomly placed “sky sources’’ across the focal plane.
For each position, the local sky flux was compared to the model prediction.
Across the sample of real LSSTCam data, the precision of the sky brightness determination was better than 1\% in approximately 99\% of all measurements, meeting the system requirement (Fig.~\ref{fig:sky_backgrounds}).

These tests used data from weekly DRP processing runs (e.g., week~18) that employed the full-focal-plane background modeling introduced with the \texttt{SkyCorrectionTask}.
This task, initially developed for HSC, was commissioned for LSSTCam and now fits large-scale gradients across the entire focal plane.
Comparisons between per-detector and full-focal-plane background solutions show that the latter reduces over-subtraction near bright, extended sources and improves uniformity across the field.

\begin{figure}[h]
    \centering
    \includegraphics[width=0.4\textwidth]{figures/sky_sources.png}

    \caption{Performance of sky brightness determination using forced photometry of randomly placed sky sources.
    Nearly all measurements ($\sim$99\%) achieve better than 1\% precision, satisfying the system requirement.}
    \label{fig:sky_backgrounds}
\end{figure}

\subsection{Limiting Surface Brightness Sensitivity}

Random sampling of sky regions from ten representative $g$-band LSSTCam visits yields a limiting surface brightness of $\sim$28~mag~arcsec$^{-2}$.
This limit is defined as three times the standard deviation of the background flux distribution within 10~arcsecond boxes.
Further validation using radial photometry of the brightest cluster galaxy (BCG) in Abell~360 confirms that surface brightness features are detected to similar levels in multiple bands.
These measurements demonstrate that the system is achieving the expected depth for extended, low surface brightness structures (Fig.~\ref{fig:lsb_results}).

\begin{figure}[h]
    \centering
    \includegraphics[width=0.4\textwidth]{figures/surfacebrightness1.png}
    \includegraphics[width=0.4\textwidth]{figures/surfacebrightness2.png}
    \caption{Low surface brightness performance.
    \textit{Left:} Random sampling of $g$-band LSSTCam visits yields a limiting surface brightness of $\sim$28~mag~arcsec$^{-2}$.
    \textit{Right:} Radial photometry around the BCG in Abell~360 confirms consistent surface brightness sensitivity across filters.}
    \label{fig:lsb_results}
\end{figure}

\subsection{Optical Ghost Area Impact}

The contribution of optical ghosts to the effective background and usable imaging area was evaluated using ray-tracing simulations with a \texttt{Batoid} optical model calibrated to the observed data.

\subsubsection{Simulation Scheme}
For each LSSTCam visit, all bright stars in the field were queried from the Yale Bright Star Catalog, and their magnitudes were transformed from V-band to LSST magnitudes using a set of transformation equations derived from the MONSTER reference catalog.
%\citep[DMTN-277]{DMTN-277}
These transformations used the V-band magnitude and B-V color of the stars. They were then initialized into a Batoid ray-tracing simulation with the reflectances of the simulated optical elements set to the values produced by systems engineering simulations in the {\tt syseng\_throughputs} package.

The rays were propagated through the full system to produce simulated optical ghosts. The ghost flux was binned at the detector level. A ghost was considered to be "impacting" if the surface brightness of the ghost (in counts per pixel) was greater than $1/3$ of the median sky noise (estimated from the data for each visit) in at least one of the detectors. The procedure is shown in detail in \citet{SITCOMTN-173}.


\begin{figure}[h]
    \centering
    \includegraphics[width=0.5\textwidth]{figures/ghost_impact.png}
    \caption{(Right) Example of an LSSTCam with ghosting in the bottom-right corner of the focal plane. (Left) Area marked as impacted highlighted in yellow using a {\tt Batoid} simulation of the bright star in the image.}
    \label{fig:ghost_impacted_area_example}
\end{figure}

\subsubsection{Impacted Area Statistics}
The week 37 DRP dataset was used as a proxy to assess the expected area impacted by optical ghosts in LSST. The DRP contained $\sim$3100 visits on which the scheme in the previous section was implemented. The top panel of Fig.~\ref{fig:ghosts} shows that the "usual case" average ghost-affected area is $\sim$0.6\%. The average ghost-affected area is highest in $u$ and lowest in $r$ and $i$. This satisfies the system requirement that the fractional area impacted by ghosts remain below 1\%.

This number, however, is highly dependent on the fields being observed. To assess the variation in the ghost-affected area, 20 highly-ghosted visits were picked in each band (except $y$) by visual inspection which serve as the "worst case" impacted area, which is highest in $u$ at $\sim$8\%.

The impacted area was also calculated as a function of star magnitude in each band by placing simulated stars at different off-axis positions within the focal plane. The bottom-right panel in Fig.~\ref{fig:ghosts} shows that the impacted-area doesn't depend strongly on the field position.


\begin{figure}[h]
    \centering
    \includegraphics[width=0.5\textwidth]{figures/ghost_impact1.png}
    \includegraphics[width=0.9\textwidth]{figures/ghost_impact2.png}
    \caption{(Top) Fraction of the focal plane impacted by optical ghosts simulated with \texttt{Batoid} as a function of bandpass. The "usual case" impacted area was calculated using DRP visits while the "worst case" impacted area was calculated using visits that contained high amounts of ghosting selected by visual inspection.
    The average impacted area is $\sim$0.6\%, with the most area lost in $u$ and least in $r$ and $i$, satisfying the requirement that ghost area loss remain below 1\%. (Bottom Left) Impacted area as a function of simulated star magnitude in each band. (Bottom Right) Impacted area as a function of field position of the simulated star in r-band. The impacted area does not vary heavily with field position.}
    \label{fig:ghosts}
\end{figure}

\subsection{Summary}

The commissioning campaign demonstrated that LSSTCam background modeling meets or exceeds requirements.
Sky brightness precision is better than 1\%, limiting surface brightness sensitivity reaches $\sim$28~mag~arcsec$^{-2}$, and the fractional area affected by optical ghosts remains below 1\%.
These results validate both the photometric and low surface brightness performance of the system and establish a foundation for continued optimization of background modeling in future data releases.