\subsection{Photometry}
\label{sec:photometry}

\begin{itemize}
    \item Figure: histogram of photometric repeatability for ensemble of visits; panel for each band?
    \item Figure: illumination correction
    \item Figure: average photometric residuals in focal plane coordinates for ensemble of visits; maybe a two-panel figure to show full focal plane and an individual detector
    \item Any other correlations of photometry that are worth exploring further (e.g., residuals w/ respect to stellar color, stellar flux, airmass)
\end{itemize}

Photometric calibration establishes the flux scale uniformity across the focal plane and over time, providing the foundation for all downstream science measurements.
The LSSTCam system now achieves internal photometric repeatability at or below the 5~millimag (mmag) level, consistent with design requirements.

\subsubsection{Reference Catalog and Initial Zeropoints}

The first stage of the calibration uses the MONSTER reference catalog \citep[DMTN-277]{DMTN-277}, a composite all-sky dataset cross-calibrated from SkyMapper, Pan-STARRS, Gaia~XP, and other surveys.
Each LSSTCam detector receives an individual zeropoint solution by matching instrumental fluxes to the reference catalog.
This per-detector calibration already delivers sub-percent photometric repeatability in the $griz$ bands, as shown in
These initial zeropoints are used for prompt processing, which does not include a global calibration step.

\begin{figure}[h]
    \centering
     \includegraphics[width=\textwidth]{figures/PA1.png}
    \caption{Per-detector zeropoint calibration using the MONSTER reference catalog.
    The preliminary calibration achieves better than 1\% repeatability for bright, isolated sources in the $griz$ bands.}
    \label{fig:perdetector_zeropoint}
\end{figure}

\subsubsection{Global Forward Calibration}

The Forward Global Calibration Method (\texttt{fgcmcal}) combines all visits to produce a self-consistent photometric solution that includes illumination and chromatic corrections.
This method jointly fits the atmosphere, instrumental throughput, and detector response across the focal plane.
The resulting photometric repeatability is 4–5~mmag in all bands ($ugrizy$), meeting the LSST photometric uniformity requirement (Fig.~\ref{fig:fgcmcal_performance}).

\begin{figure}[h]
    \centering
    \includegraphics[width=0.6\textwidth]{figures/fgcmcal_performance.png}
    \caption{Photometric repeatability before and after application of the Forward Global Calibration Method (\texttt{fgcmcal}).
    The global solution achieves 4–5~mmag internal repeatability across all bands.}
    \label{fig:fgcmcal_performance}
\end{figure}

At this precision, the observed scatter is approaching the limit set by Poisson noise in the stellar measurements themselves.
The intrinsic calibration precision is therefore likely closer to the 2~mmag level.

\subsection{Chromatic Response Across the Focal Plane}

The chromatic response of the LSSTCam focal plane varies with both detector type and filter transmission.
The effect is strongest in the $g$~band, where the hybrid focal plane combines ITL and e2v sensors with slightly different quantum efficiency curves.
Figure~\ref{fig:chromatic_response} compares the predicted chromatic response—based on laboratory filter scans and detector QE measurements—to the measured on-sky response derived from stellar photometry.

\begin{figure}[h]
    \centering
    \includegraphics[width=0.45\textwidth]{figures/chromatic_response_predicted.png}
     \includegraphics[width=0.45\textwidth]{figures/chromatic_response_measured.png}
    \caption{Predicted (left) and measured (right) chromatic response across the $g$-band focal plane.
    The measured variation is within $\pm20$~mmag, demonstrating excellent agreement with the predicted response.
    Accurate modeling of this chromatic structure is critical for maintaining uniformity at the sub-percent level.}
    \label{fig:chromatic_response}
\end{figure}

The strong correspondence between the measured and predicted patterns indicates that the physical modeling of the detector and filter system is accurate at the level required for precision photometry.
Incorporating these chromatic corrections into \texttt{fgcmcal} ensures that color-dependent throughput effects are accounted for in the final photometric solution.

\subsubsection{Summary}

LSSTCam photometric calibration now achieves internal repeatability at or below 5~mmag, comfortably meeting design requirements.
Per-detector zeropoints using the MONSTER catalog already reach sub-percent precision, while the global \texttt{fgcmcal} solution provides 4–5~mmag repeatability across all bands.
Measured chromatic response maps agree closely with predictions, validating the laboratory throughput models and confirming the accuracy of the chromatic correction applied in the pipeline.
Future work will focus on refining illumination corrections and incorporating updated reference catalogs for the final DR1 global calibration.
