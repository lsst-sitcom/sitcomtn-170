\section{On-sky Commissioning Campaign with LSSTCam}
\label{sec:campaign}

\begin{itemize}
    \item Brief summary of the time window of the observations, basic statistics on the number of visits, and overview of the types of observations.
    \item Brief description of core datasets used for this report. Table to summarize observations.
    \item Description of First Look dataset
    \item Potentially a figure to show sky coverage (could be a representative small field) used for evaluation of System First Light milestone
\end{itemize}

\subsection{Science Program Observations}

\begin{itemize}
    \item Demonstrated Concept of Operations during 23 weeks of the on-sky campaign w/ LSSTCam
    \item Acquired $\sim$80K total visits from 15 April to 21 September, including $\sim$20K flat, $\sim$9K bias, and $\sim$5K dark exposures for in-dome calibrations, $\sim$22K on-sky engineering images primarily for Active Optics System (AOS) commissioning, and $\sim$22K images intended for Science Pipelines commissioning
    \begin{itemize}
        \item Total volume of pixel data $\sim$100 times larger than that from 7-week on-sky campaign w/ LSSTComCam
    \end{itemize}
    \item Completed observations of selected fields to support Rubin First Look media event
    \item Demonstrated  full-nights of wide-area survey-mode observations driven by Feature Based Scheduler (FBS)
     \begin{itemize}
        \item Scheduler configuration similar to LSST using input telemetry sources from the summit
     \end{itemize}
    \item Acquired core on-sky datasets for Science Pipelines commissioning, including
     \begin{itemize}
        \item observations in every band reaching 10-year LSST equivalent exposure in at least one field
        \item densely dithered star field observations in \emph{ugrizy} bands to evaluate internal calibration
        \item observations to build templates and run difference imaging analysis w/ Prompt Processing for both individual fields and SV survey Wide area
     \end{itemize}
    \item Observed Deep Drilling Fields (DDFs) using variety of dither patterns to inform LSST DDF observing strategy
    \item Demonstrated Target of Opportunity (ToO) Observations
    \item Provided regular updates for science community via weekly Commissioning Update posts, public nightly reporting (ls.st/svnightly), public summary reporting (survey-strategy.lsst.io/progress), in addition to internal survey progress monitoring tools (including NightlyDigest)
\end{itemize}

\begin{table*}
    \small
    \centering
    \begin{tabular}{ll}
        15 Apr  &		First night sky images w/ LSSTCam \\
        4 May   &		Rubin First Look observations completed as part of Small Field Surveys \\
        9 Jun   &		Start wide-area survey-mode observation engineering \\
        20 Jun	&	    Start of pilot SV Survey observations w/ $\sim$2 hours per night \\
        Early Jul & 	Multiple consecutive full nights of SV survey operations; System First Light technical milestone \\
        Late Jul-Aug &	Multiple winter storms substantially limit opportunities for on-sky observing \\
        24 Jul	&		One of the five filter sockets on LSSTCam becomes non-operational until engineering downtime \\
        Early Aug &		Test priorities shift to emphasize improvements to consistency of delivered image quality \\
        10 Aug &		FBS configuration updated to reduce footprint of Wide from 3000 deg2 to 750 deg2 \\
        12 Aug &		Last filter swap during on-sky campaign w/ LSSTCam; remaining observations use griz filters only \\
        5 Sep &			FBS configuration updated for longer DDF sequences; prioritize ECDFS and ELAIS-S1 DDFs \\
        15 Sep &		FBS configuration updated to target only regions with deployed template coverage \\
        21 Sep &		Last night of on-sky commissioning campaign w/ LSSTCam \\
        22 Sep &		Start final construction downtime and the first operations engineering downtime \\
        20 Oct &		Resumed on-sky observations; transition to Early Operations system optimization \\
    \end{tabular}
    \caption{Events during on-sky campaign w/ LSSTCam.}
    \label{tab:target_fields_pointing_centers}
\end{table*}

\begin{figure}
    \begin{center}
      \includegraphics[width=0.8\textwidth]{figures/sky_coverage.png}
    \end{center}
    \caption{Sky coverage during the LSSTCam on-sky campaign.}
    \label{fig:sky_coverage}
\end{figure}

\begin{figure}
    \begin{center}
      \includegraphics[width=1.0\textwidth]{figures/on-sky_time_sv_surveys.png}
    \end{center}
    \caption{On-sky time during SV surveys.}
    \label{fig:on-sky_time_sv_surveys}
\end{figure}

\begin{figure}
    \begin{center}
      \includegraphics[width=0.8\textwidth]{figures/area_coverage_per_band.png}
    \end{center}
    \caption{Area coverage per band.}
    \label{fig:area_coverage_per_band}
\end{figure}

\begin{figure}
    \begin{center}
      \includegraphics[width=0.8\textwidth]{figures/sv_surveys_cumulative_visit_counts.png}
    \end{center}
    \caption{SV surveys cumulative visit counts.}
    \label{fig:sv_surveys_cumulative_visit_counts}
\end{figure}

\subsubsection{Field Surveys}

Summary of pointings, distribution of visits, dither patterns

\subsubsection{Science Validations Surveys}

Some of the summary visuals can also be included in this technote

\subsection{Overall On-sky Efficiency}

Summary figures to show how the time was utilized during the on-sky campaign

Estimate of system availability during the SV surveys (could be that our best data for this purpose is from early July when we had full nights of SV)