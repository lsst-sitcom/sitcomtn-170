\section{On-sky Commissioning Campaign with LSSTCam}
\label{sec:campaign}

\begin{itemize}
    \item Brief summary of the time window of the observations, basic statistics on the number of visits, and overview of the types of observations.
    \item Brief description of core datasets used for this report. Table to summarize observations.
    \item Description of First Look dataset
    \item Potentially a figure to show sky coverage (could be a representative small field) used for evaluation of System First Light milestone
\end{itemize}

\subsection{Science Program Observations}

\begin{figure}
    \begin{center}
      \includegraphics[width=0.8\textwidth]{figures/sky_coverage.png}
    \end{center}
    \caption{Sky coverage during the LSSTCam on-sky campaign.}
    \label{fig:sky_coverage}
\end{figure}

\begin{figure}
    \begin{center}
      \includegraphics[width=1.0\textwidth]{figures/on-sky_time_sv_surveys.png}
    \end{center}
    \caption{On-sky time during SV surveys.}
    \label{fig:on-sky_time_sv_surveys}
\end{figure}

\begin{figure}
    \begin{center}
      \includegraphics[width=0.8\textwidth]{figures/area_coverage_per_band.png}
    \end{center}
    \caption{Area coverage per band.}
    \label{fig:area_coverage_per_band}
\end{figure}


\subsubsection{Field Surveys}

Summary of pointings, distribution of visits, dither patterns

\subsubsection{Science Validations Surveys}

Some of the summary visuals can also be included in this technote

\subsection{Overall On-sky Efficiency}

Summary figures to show how the time was utilized during the on-sky campaign

Estimate of system availability during the SV surveys (could be that our best data for this purpose is from early July when we had full nights of SV)