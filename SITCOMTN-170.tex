\documentclass[SE,lsstdraft,authoryear,toc]{lsstdoc}
\input{meta}

% Package imports go here.

% Local commands go here.
\newcommand{\VeraRubinObservatory}{NSF-DOE Vera~C.~Rubin Observatory\xspace}

\newcommand{\figRef}[1]{Fig.~\ref{fig:#1}\xspace}
\newcommand{\figRefII}[2]{Figs.~\ref{fig:#1} and \ref{fig:#2}\xspace}
\newcommand{\figRefIII}[3]{Figs.~\ref{fig:#1}, \ref{fig:#2}, and \ref{fig:#3}\xspace}
\renewcommand{\secRef}[1]{Sec.~\ref{sec:#1}\xspace}
\newcommand{\tabRef}[1]{Tab.~\ref{tab:#1}\xspace}

%If you want glossaries
%\input{aglossary.tex}
%\makeglossaries

\title{An Interim Report on the On-Sky Commissioning Campaign with LSSTCam}

% This can write metadata into the PDF.
% Update keywords and author information as necessary.
\hypersetup{
    pdftitle={An Interim Report on the On-Sky Commissioning Campaign with LSSTCam},
    pdfauthor={NSF-DOE Vera C. Rubin Observatory},
    pdfkeywords={}
}

% Optional subtitle
% \setDocSubtitle{A subtitle}

\input{authors}

\setDocRef{SITCOMTN-170}
\setDocUpstreamLocation{\url{https://github.com/lsst-sitcom/sitcomtn-170}}

\date{\vcsDate}

% Optional: name of the document's curator
% \setDocCurator{The Curator of this Document}

\setDocAbstract{%
From 15 April to 21 September 2025, the \VeraRubinObservatory conducted on an on-sky commissioning campaign using the LSST Camera (LSSTCam) to test the end-to-end functionality of hardware and software, as well as operational procedures.
This interim report provides a preliminary technical overview of our understanding of the integrated system performance based tests and analyses conducted during the on-sky commissioning campaign with LSSTCam.
The objectives are to synthesize what we have learned about the system in a timely way to inform Early Operations optimization, and to inform the Rubin science community on the progress of the LSSTCam on-sky campaign.
The report is organized into sections that describe major activities during the campaign, as well as multiple aspects of the demonstrated system and science performance.
All of the results presented here are to be understood as work in progress using engineering data and the initial versions of the data processing pipelines; the report is a living document that will be
updated as analyses are refined.
}

% Change history defined here.
% Order: oldest first.
% Fields: VERSION, DATE, DESCRIPTION, OWNER NAME.
% See LPM-51 for version number policy.
\setDocChangeRecord{%
  \addtohist{1}{YYYY-MM-DD}{Unreleased.}{Bechtol}
}


\begin{document}

% Create the title page.
\maketitle
% Frequently for a technote we do not want a title page  uncomment this to remove the title page and changelog.
% use \mkshorttitle to remove the extra pages

% ADD CONTENT HERE
% You can also use the \input command to include several content files.

\section{Introduction}
\label{sec:introduction}

The \VeraRubinObservatory on-sky commissioning campaign using the LSST Camera (hereafter LSSTCam) began on 15 April 2025 and ended on 21 September 2025.
This interim report provides a concise summary of our understanding of the integrated system performance based tests and analyses conducted during the LSSTCam on-sky campaign.
We seek to distill, and to communicate in a timely way, what we have learned about the system to support the transition from Rubin Observatory Construction to Operations.
The report is organized into sections that describe major activities during the campaign, as well as multiple aspects of the demonstrated system and science performance.

\begin{warning}[Preliminary Results]
All of the results presented here are to be understood as work in progress using engineering data and the
initial versions of the data processing pipelines.
It is expected at this stage, immediately following the completion of the on-sky commissioning campaign, that several analyses are still in progress, and that some of the discussion will concern open questions, issues, and anomalies that are actively being worked by the team to enhance the system reliability.
Additional documentation will be provided as our understanding of the demonstrated performance of the as-built system progresses.
\end{warning}

\subsection{Charge}

\begin{note}[Charge Development Historical Note]
    The initial version of the charge developed in September 2025 is provided below for reference.
\end{note}

We identify the following high-level goals for this interim report:

\begin{itemize}

    \item \textbf{Document our current understanding of the integrated system performance} to support systems engineering verification activities associated with demonstrating Construction Completeness \citedsp{sitcomtn-005}.

    \item \textbf{Transfer knowledge to support the transition from Construction to Operations} to inform the Early Operations optimization period and to support the Early Science Program \citeds{RTN-011}.

    \item \textbf{Inform the Rubin Science Community} on the progress of the on-sky commissioning campaign using LSSTCam.

\end{itemize}

Formal acceptance testing with respect to system-level requirement specifications (\citedsp{lse-29} and \citedsp{lse-30}) will be recorded using the LSST Verification \& Validation (LVV) system.
By design, several of the analyses presented in this report correspond to system-level requirements, and therefore, this report is anticipated to serve as a verification artifact to support several of those systems engineering activities.

\textbf{The groups within the Rubin Observatory project working on each of the activities and performance analyses are charged with contributing to the relevant sections of the report.}
The anticipated level of detail for the sections ranges from a paragraph up to a page or two of text, depending on the current state of understanding, with \textbf{quantitative performance} expressed as summary statistics, tables, and/or figures.
The objective for this document is to \textbf{summarize the state of knowledge of the system}, rather than how we got there or ``lessons learned''.
The sections refer to additional supporting documentation, e.g., analysis notebooks, other technotes with further detail, as needed.
Given the timelines for commissioning various aspects of the system, it is natural that some sections will have more detail than others.

The anticipated milestones for developing this interim report are as follows:

\begin{itemize}

    \item 18 Sep 2025: Define charge

    \item 22 Sep 2025: On-sky commissioning campaing with LSSTCam completed; start of final construction downtime and its first operations engineering downtime

    \item 8 Oct 2025: Detailed outlines with essential figures and performance statistics for report sections made available for internal review (can be on unmerged development branches)

    \item 15 Oct 2025: Revised drafts of report sections made available for internal review; development branches merged to main branch; editing for consistency and coherency throughout the report

    \item 22 Oct 2025: Start of Construction to Operations Transition Workshop; advanced draft ready for review by full Rubin Observatory team

    \item 31 Oct 2025: Initial version of report is released

\end{itemize}

\section{Executive Summary}
\label{sec:summary}

Executive summary here.

\begin{note}[Versioning Note]
    This interim report provides a preliminary technical overview of the LSSTCam on-sky campaign based on analyses through October 2025.
\end{note}

\subsection{Accomplishments}

\begin{itemize}
    \item \textbf{Accomplishment.} Description.
    \item \textbf{Accomplishment.} Description.
    \item \textbf{Accomplishment.} Description.
\end{itemize}

\subsection{Areas of Ongoing Investigation and Further Development}

\begin{itemize}
    \item \textbf{Issue.} Description.
    \item \textbf{Issue.} Description.
    \item \textbf{Issue.} Description.
\end{itemize}

\section{On-sky Commissioning Campaign with LSSTCam}
\label{sec:campaign}

\begin{itemize}
    \item Brief summary of the time window of the observations, basic statistics on the number of visits, and overview of the types of observations.
    \item Brief description of core datasets used for this report. Table to summarize observations.
    \item Description of First Look dataset
    \item Potentially a figure to show sky coverage (could be a representative small field) used for evaluation of System First Light milestone
\end{itemize}

\section{Calibration Systems}
\label{sec:calibration_systems}

\begin{figure}
    \begin{center}
      \includegraphics[width=0.8\textwidth]{figures/single-led_flatfield_screen.png}
    \end{center}
    \caption{Single-LED flatfield screen.}
    \label{fig:single-led}
\end{figure}



\begin{itemize}
    \item Calibration Data without external illumination
    \begin{itemize}
        \item Biases
        \item Darks
    \end{itemize}
    \item Calibration System Sources
    \begin{itemize}
        \item White Light (LED) w/ Fiber Spectrograph and Photodiode + Electrometer
        \item Tunable Laser w/ Fiber Spectrograph and Photodiode + Electrometer
    \end{itemize}
    \item Flat Field Projector system
    \begin{itemize}
        \item (Single) LED flats
        \item Photon Transfer Curve
    \end{itemize}
    \item Collimated Beam Projector (CBP)
    \begin{itemize}
        \item Filter scans (and no-filter scans)
        \item Crosstalk spots
    \end{itemize}
\end{itemize}


\section{Sensor Performance and Instrument Signature Removal}
\label{sec:camera}

\begin{itemize}
    \item Usable pixels, effective field of view, fill factor
    \item Read noise
    \item Crosstalk
    \item Dynamic range (brightest and faintest objects)
    \item Summary table of key Camera performance metrics
    \item Section needs at least one figure to visualize focal plane; maybe a flat?
    \item Standard visit (i.e., snaps) evaluation?
\end{itemize}

A total of 188/189 sensors are operational, with only R30/S12 currently non-operational.
The median amplifier loses $\sim$0.45\% of its area to defects, while the mean across amplifiers is closer to $\sim$0.7\%.
Elevated defect counts appear where effective QE is lower at detector edges, especially for Cy0 and Cy7 amplifiers.
No new static defect classes have been identified on e2v devices (hot/cold pixels and columns remain as expected), while a subset of ITL detectors exhibits “vampire” pixels and locally higher noise.
Dynamic masking still requires work, including treatment of e2v edge bleeds and occasional flare-type artifacts when bright stars fall on extreme detector edges.
Overall, $\gtrsim$99\% of the focal plane is active and usable within the calibrated field.

\begin{figure}[h]
    \centering
    \includegraphics[width=\textwidth]{figures/usable_pixels.png}
    \caption{Usable pixels summary: 188/189 sensors operational (R30/S12 offline), median amplifier defect area $\sim$0.45\% (mean $\sim$0.7\%), with edge-related QE behavior raising counts in Cy0/Cy7 amplifiers; no new static defect types on e2v, “vampire” pixels seen on some ITL detectors.}
    \label{fig:usable_pixels}
\end{figure}

Optical vignetting limits the fully calibrate-able radius to the camera requirement of 317~mm (black circle).
We introduced a \texttt{PARTLY\_VIGNETTED} mask plane for 317–350~mm radius (red circle), where calibration is expected to be feasible but at reduced fidelity relative to the inner zone.
If calibration to 350~mm is achieved, the residual loss of area relative to the geometric maximum radius of 365~mm (blue circle) is minimal, affecting only the extreme corners of the focal plane.
Regions outside 350~mm are treated as fully vignetted and excluded from science products.

\begin{figure}[h]
    \centering
    \includegraphics[width=\textwidth]{figures/vignetting_fov.png}
    \caption{Optical vignetting and calibrate-able field of view: requirement radius 317~mm (black), target extension to 350~mm (red) for partial vignetting, and geometric maximum 365~mm (blue).}
    \label{fig:vignetting_fov}
\end{figure}

Read-noise maps show all channels in specification, with ITL devices having slightly higher medians than e2v and some top–bottom amplifier shifts visible on subsets of ITL sensors.
Gain maps show nominal values consistent with design and optimal ADC usage.
PTC turn-off levels indicate amplifier dynamic ranges typically near $\sim$$10^5$\,DN, with subsets lower; pixels above the turn-off are saturated in processing.
CBP spot data confirm that the updated sequencer reduces noise and correlations, but crosstalk coefficients require re-derivation post-change; first-order terms are largely removed while higher orders remain to be tuned for DP2 processing.
No new major anomalies were uncovered during these campaigns, and calibration improvements are being integrated iteratively into ISR and downstream processing.

\subsection{Summary}

Operational yield is 188/189 sensors with $\gtrsim$99\% active focal-plane area within the calibrate-able region. :contentReference[oaicite:4]{index=4}
A \texttt{PARTLY\_VIGNETTED} mask enables use out to 350~mm radius with controlled fidelity, minimizing area loss to extreme corners while avoiding over-fitting in fully vignetted zones.
Read noise, gain, and dynamic range meet expectations, and post-sequencer crosstalk calibration is in progress for DP2.


\subsection{Bright Stars and Moon}

On-sky validation that the Camera can survey the night sky in the presence of bright sources

\subsection{Sensor Anomalies}

Any sensor anomalies that are worth noting

\subsection{LSSTCam Performance during the LSSTCam On-sky Campaign}

Brief summary of LSSTCam performance during the campaign, any open questions, outstanding issues

This could include any discussion on Camera susbsystems, e.g., focal plane optimization, filter exchange system, cryo, camera shutter, that should be highlighted

\section{System Optical Throughput for Focused Light}
\label{sec:throughput}

\begin{itemize}
    \item Standard bandpass; includes the sensors, filters, lenses, mirrors, and (a standard) atmosphere. Measured with CBP? Monochromatic flats with flat field screen?
    \item Imaging depth in multiple bands (LSR-REQ-0090); also express as zeropoint to separate out the effects of image quality; could be comparison to refcats and/or spectrophotometric standards
    \item Figure to show throughput variation of throughput across field of view (LSR-REQ-0109); potentially separating out vignetting and CCD response
    \item Discussion on Sensitivity Factor (fS) in the System Performance Diagram
\end{itemize}

\subsection{Standard Bandpass}

\subsection{Measured Zeropoints}

\begin{figure}
    \begin{center}
      \includegraphics[width=0.8\textwidth]{figures/c26202.png}
    \end{center}
    \caption{C26202.}
    \label{fig:c26202}
\end{figure}

\begin{figure}
    \begin{center}
      \includegraphics[width=0.8\textwidth]{figures/zeropoint_cloud_extinction.png}
    \end{center}
    \caption{Zeropoint cloud extinction.}
    \label{fig:zeropoint_cloud_extinction}
\end{figure}


\section{Measured Sky Backgrounds}
\label{sec:sky_background}

During the LSSTCam on-sky campaign, dedicated tests were performed to characterize the background, evaluate the performance of the \texttt{SkyCorrectionTask}, and verify compliance with project-level requirements.

\subsection{Sky Brightness Determination}

Sky brightness measurements were performed using forced photometry on randomly placed “sky sources’’ across the focal plane.
For each position, the local sky flux was compared to the model prediction.
Across the sample of real LSSTCam data, the precision of the sky brightness determination was better than 1\% in approximately 99\% of all measurements, meeting the system requirement (Fig.~\ref{fig:sky_backgrounds}).

These tests used data from weekly DRP processing runs (e.g., week~18) that employed the full-focal-plane background modeling introduced with the \texttt{SkyCorrectionTask}.
This task, initially developed for HSC, was commissioned for LSSTCam and now fits large-scale gradients across the entire focal plane.
Comparisons between per-detector and full-focal-plane background solutions show that the latter reduces over-subtraction near bright, extended sources and improves uniformity across the field.

\begin{figure}[h]
    \centering
    \includegraphics[width=0.4\textwidth]{figures/sky_sources.png}

    \caption{Performance of sky brightness determination using forced photometry of randomly placed sky sources.
    Nearly all measurements ($\sim$99\%) achieve better than 1\% precision, satisfying the system requirement.}
    \label{fig:sky_backgrounds}
\end{figure}

\subsection{Limiting Surface Brightness Sensitivity}

Random sampling of sky regions from ten representative $g$-band LSSTCam visits yields a limiting surface brightness of $\sim$28~mag~arcsec$^{-2}$.
This limit is defined as three times the standard deviation of the background flux distribution within 10~arcsecond boxes.
Further validation using radial photometry of the brightest cluster galaxy (BCG) in Abell~360 confirms that surface brightness features are detected to similar levels in multiple bands.
These measurements demonstrate that the system is achieving the expected depth for extended, low surface brightness structures (Fig.~\ref{fig:lsb_results}).

\begin{figure}[h]
    \centering
    \includegraphics[width=0.4\textwidth]{figures/surfacebrightness1.png}
    \includegraphics[width=0.4\textwidth]{figures/surfacebrightness2.png}
    \caption{Low surface brightness performance.
    \textit{Left:} Random sampling of $g$-band LSSTCam visits yields a limiting surface brightness of $\sim$28~mag~arcsec$^{-2}$.
    \textit{Right:} Radial photometry around the BCG in Abell~360 confirms consistent surface brightness sensitivity across filters.}
    \label{fig:lsb_results}
\end{figure}

\subsection{Optical Ghost Area Impact}

The contribution of optical ghosts to the effective background and usable imaging area was evaluated using ray-tracing simulations with the \texttt{Batoid} optical model.
Simulated stars of varying brightness were placed at different field positions, and the fractional area of the focal plane impacted by ghost reflections was measured as a function of bandpass.
The mean ghost-affected area across all filters is $\sim$0.6\%, with the largest values in $u$ and the smallest in $r$ and $i$.
This satisfies the system requirement that the fractional area impacted by ghosts remain below 1\%.
On-sky verifications confirm that the simulated ghost patterns correspond well to observed behavior (Fig.~\ref{fig:ghosts}).

\begin{figure}[h]
    \centering
    \includegraphics[width=0.5\textwidth]{figures/ghost_impact1.png}
    \includegraphics[width=0.9\textwidth]{figures/ghost_impact2.png}
    \caption{Fraction of the focal plane impacted by optical ghosts simulated with \texttt{Batoid}.
    The mean loss is $\sim$0.6\% across all filters, with worst performance in $u$ and best in $r$ and $i$, satisfying the requirement that ghost area loss remain below 1\%.}
    \label{fig:ghosts}
\end{figure}

\subsection{Summary}

The commissioning campaign demonstrated that LSSTCam background modeling meets or exceeds requirements.
Sky brightness precision is better than 1\%, limiting surface brightness sensitivity reaches $\sim$28~mag~arcsec$^{-2}$, and the fractional area affected by optical ghosts remains below 1\%.
These results validate both the photometric and low surface brightness performance of the system and establish a foundation for continued optimization of background modeling in future data releases.

\section{Delivered Image Quality}
\label{sec:image_quality}

Section Editor: (?)

\begin{itemize}
    \item Figure: distribution of PSF FWHM for an ensemble of visits; expect that we will want a comparison to the atmosphere seeing in order to estimate the system contribution
    \item Figure: average PSF size and ellipticity distribution across field of view for an ensemble of visits
    \item Figure: measured PSF size as a function of magnitude
    \item Figure: measured PSF size as a function of stellar color
    \item Figure: wings of PSF and encircled energy as a function of radius
    \item Brief discussion of current assessment of the image quality budget (expect that details of quantifying various contributions to the image quality budget will be discussed elsewhere?)
\end{itemize}

\section{Stray and Scattered Light}
\label{sec:stray_light}

The unique, wide-field design of the Simonyi Survey Telescope it particularly susceptible to stray and scattered light. Since LSST seeks to investigate the low-surface-brightness universe \citep{2019ApJ...873..111I}, it is important to identify, model, and mitigate stray-light artifacts. This becomes particularly important in the context of the  the delayed installation of the light--wind screen \citep[LWS;][]{Marchiori:2024}. A general overview of the investigation of stray light features during commissioning can be found in \citet{SITCOMTN-160}, while a detailed investigation of the most prominent stray light feature (the ``scratched tape'') is discussed in \citet{SITCOMTN-166}. Here, we provide a brief summary of those efforts and the impact of stray and scattered light.


\begin{itemize}
  %ADW: Ghosts should be discussed elsewhere
  %\item Figure: examples of ghosts; are the features understood?
  \item Figure: examples of stray and scattered light; are the sources of stray and scattered light understood?
  \item Summative discussion on the impacts of stray and scattered light; how prevalent, amplitude and structure of the features, to what extent will additional baffling (e.g., with LWS) mitigate the features
\end{itemize}


\section{System Timing and Dynamics}
\label{sec:system_timing}

\begin{itemize}
    \item Standard Visit Duration (OSS-REQ-0288)
    \item Readout time – discussed with the Camera?
    \item Time Interval Between Visits (OSS-REQ-0289)
    \item Maximum time for operational filter change (OSS-REQ-0293)
    \item Telescope Azimuth Slewing Rate (TLS-REQ-0029)
    \item Telescope Elevation Slewing Rate (TLS-REQ-0159)
    \item Summative assessment on rate of acquiring observations
\end{itemize}

\subsection{Standard Visit Definition}

Discussion on decision to use 30-second exposures

\subsection{Visit Timing and Interval between Visits}

Camera readout time, filter change times

Telescope motion settings, slew and setttle, distribution of time between visits

\subsection{Effective Survey Speed}

Observing efficiency factor (fO) for System Performance diagram

Survey simulations combined with telescope motion capabilities; compare with actual rate of acquiring visits during SV surveys


\section{Data Management}
\label{sec:data_management}

Section Editor: (?)

\begin{itemize}
    \item Brief description (paragraph or two) of data processing campaigns during on-sky commissioning, mainly reporting on the functional capabilities; algorithms and data products are discussed elsewhere; pointers to other references
    \item Figure with representative pixel-level color coadd image?
\end{itemize}

\section{Calibration}
\label{sec:calibration}

\subsection{PSF Characterization}
\label{sec:psf}

\begin{itemize}
    \item Figure: PSF residuals across the field of view
    \item Figure: measured PSF size as a function of magnitude (could include brighter-fatter discussion)
    \item Figure: measured PSF size as a function of stellar color
    \item Figure: wings of PSF and encircled energy as a function of radius
\end{itemize}

\begin{figure}
    \begin{center}
      \includegraphics[width=0.45\textwidth]{figures/psf_size_residuals_focal_plane.png}
      \includegraphics[width=0.45\textwidth]{figures/psf_height_map.png}
    \end{center}
    \caption{PSF size residuals and height map.}
    \label{fig:psf_size_residuals}
\end{figure}

\begin{figure}
    \begin{center}
      \includegraphics[width=0.8\textwidth]{figures/psf_brighter_fatter.png}
    \end{center}
    \caption{Brighter fatter.}
    \label{fig:brighter_fatter}
\end{figure}

\subsection{Astrometry}
\label{sec:astrometry}

\begin{itemize}
    \item Figure: histogram of astrometric repeatability for ensemble of visits
    \item Figure: static camera astrometric distortion model
    \item Figure: average astrometric residuals in focal plane coordinates for ensemble of visits; two-panel figure to show full focal plane and an individual detector
    \item Figure: average E/B mode across ensemble of visits; expect to see mostly a pure E-mode astrometric field indicative of residuals dominated by astrometric turbulence
\end{itemize}

\subsection{Photometry}
\label{sec:photometry}

\begin{itemize}
    \item Figure: histogram of photometric repeatability for ensemble of visits; panel for each band?
    \item Figure: illumination correction
    \item Figure: average photometric residuals in focal plane coordinates for ensemble of visits; maybe a two-panel figure to show full focal plane and an individual detector
    \item Any other correlations of photometry that are worth exploring further (e.g., residuals w/ respect to stellar color, stellar flux, airmass)
\end{itemize}

Photometric calibration establishes the flux scale uniformity across the focal plane and over time, providing the foundation for all downstream science measurements.
The LSSTCam system now achieves internal photometric repeatability at or below the 5~millimag (mmag) level, consistent with design requirements.

\subsubsection{Reference Catalog and Initial Zeropoints}

The first stage of the calibration uses the MONSTER reference catalog \citep[DMTN-277]{DMTN-277}, a composite all-sky dataset cross-calibrated from SkyMapper, Pan-STARRS, Gaia~XP, and other surveys.
Each LSSTCam detector receives an individual zeropoint solution by matching instrumental fluxes to the reference catalog.
This per-detector calibration already delivers sub-percent photometric repeatability in the $griz$ bands, as shown in
These initial zeropoints are used for prompt processing, which does not include a global calibration step.

\begin{figure}[h]
    \centering
     \includegraphics[width=\textwidth]{figures/PA1.png}
    \caption{Per-detector zeropoint calibration using the MONSTER reference catalog.
    The preliminary calibration achieves better than 1\% repeatability for bright, isolated sources in the $griz$ bands.}
    \label{fig:perdetector_zeropoint}
\end{figure}

\subsubsection{Global Forward Calibration}

The Forward Global Calibration Method (\texttt{fgcmcal}) combines all visits to produce a self-consistent photometric solution that includes illumination and chromatic corrections.
This method jointly fits the atmosphere, instrumental throughput, and detector response across the focal plane.
The resulting photometric repeatability is 4–5~mmag in all bands ($ugrizy$), meeting the LSST photometric uniformity requirement (Fig.~\ref{fig:fgcmcal_performance}).

\begin{figure}[h]
    \centering
    \includegraphics[width=0.6\textwidth]{figures/fgcmcal_performance.png}
    \caption{Photometric repeatability before and after application of the Forward Global Calibration Method (\texttt{fgcmcal}).
    The global solution achieves 4–5~mmag internal repeatability across all bands.}
    \label{fig:fgcmcal_performance}
\end{figure}

At this precision, the observed scatter is approaching the limit set by Poisson noise in the stellar measurements themselves.
The intrinsic calibration precision is therefore likely closer to the 2~mmag level.

\subsection{Chromatic Response Across the Focal Plane}

The chromatic response of the LSSTCam focal plane varies with both detector type and filter transmission.
The effect is strongest in the $g$~band, where the hybrid focal plane combines ITL and e2v sensors with slightly different quantum efficiency curves.
Figure~\ref{fig:chromatic_response} compares the predicted chromatic response—based on laboratory filter scans and detector QE measurements—to the measured on-sky response derived from stellar photometry.

\begin{figure}[h]
    \centering
    \includegraphics[width=0.45\textwidth]{figures/chromatic_response_predicted.png}
     \includegraphics[width=0.45\textwidth]{figures/chromatic_response_measured.png}
    \caption{Predicted (left) and measured (right) chromatic response across the $g$-band focal plane.
    The measured variation is within $\pm20$~mmag, demonstrating excellent agreement with the predicted response.
    Accurate modeling of this chromatic structure is critical for maintaining uniformity at the sub-percent level.}
    \label{fig:chromatic_response}
\end{figure}

The strong correspondence between the measured and predicted patterns indicates that the physical modeling of the detector and filter system is accurate at the level required for precision photometry.
Incorporating these chromatic corrections into \texttt{fgcmcal} ensures that color-dependent throughput effects are accounted for in the final photometric solution.

\subsubsection{Summary}

LSSTCam photometric calibration now achieves internal repeatability at or below 5~mmag, comfortably meeting design requirements.
Per-detector zeropoints using the MONSTER catalog already reach sub-percent precision, while the global \texttt{fgcmcal} solution provides 4–5~mmag repeatability across all bands.
Measured chromatic response maps agree closely with predictions, validating the laboratory throughput models and confirming the accuracy of the chromatic correction applied in the pipeline.
Future work will focus on refining illumination corrections and incorporating updated reference catalogs for the final DR1 global calibration.


\appendix

\section{Acknowledgements}

This material is based upon work supported in part by the National Science Foundation through Cooperative Agreements AST-1258333 and AST-2241526 and Cooperative Support Agreements AST-1202910 and AST-2211468 managed by the Association of Universities for Research in Astronomy (AURA), and the Department of Energy under Contract No.\ DE-AC02-76SF00515 with the SLAC National Accelerator Laboratory managed by Stanford University.
Additional Rubin Observatory funding comes from private donations, grants to universities, and in-kind support from LSST-DA Institutional Members.

% Include all the relevant bib files.
% https://lsst-texmf.lsst.io/lsstdoc.html#bibliographies
\section{References} \label{sec:bib}
\renewcommand{\refname}{} % Suppress default Bibliography section
\bibliography{local,lsst,lsst-dm,refs_ads,refs,books}

% Make sure lsst-texmf/bin/generateAcronyms.py is in your path
\section{Acronyms} \label{sec:acronyms}
\input{acronyms.tex}
% If you want glossary uncomment below -- comment out the two lines above
%\printglossaries





\end{document}
