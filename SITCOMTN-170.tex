\documentclass[SE,lsstdraft,authoryear,toc]{lsstdoc}
\input{meta}

% Package imports go here.

% Local commands go here.
\newcommand{\VeraRubinObservatory}{NSF-DOE Vera~C.~Rubin Observatory\xspace}

\newcommand{\figRef}[1]{Fig.~\ref{fig:#1}\xspace}
\newcommand{\figRefII}[2]{Figs.~\ref{fig:#1} and \ref{fig:#2}\xspace}
\newcommand{\figRefIII}[3]{Figs.~\ref{fig:#1}, \ref{fig:#2}, and \ref{fig:#3}\xspace}
\renewcommand{\secRef}[1]{Sec.~\ref{sec:#1}\xspace}
\newcommand{\tabRef}[1]{Tab.~\ref{tab:#1}\xspace}

%If you want glossaries
%\input{aglossary.tex}
%\makeglossaries

\title{An Interim Report on the On-Sky Commissioning Campaign with LSSTCam}

% This can write metadata into the PDF.
% Update keywords and author information as necessary.
\hypersetup{
    pdftitle={An Interim Report on the On-Sky Commissioning Campaign with LSSTCam},
    pdfauthor={NSF-DOE Vera C. Rubin Observatory},
    pdfkeywords={}
}

% Optional subtitle
% \setDocSubtitle{A subtitle}

\input{authors}

\setDocRef{SITCOMTN-170}
\setDocUpstreamLocation{\url{https://github.com/lsst-sitcom/sitcomtn-170}}

\date{\vcsDate}

% Optional: name of the document's curator
% \setDocCurator{The Curator of this Document}

\setDocAbstract{%
From 15 April to 21 September 2025, the \VeraRubinObservatory conducted on an on-sky commissioning campaign using the LSST Camera (LSSTCam) to test the end-to-end functionality of hardware and software, as well as operational procedures.
This interim report provides a preliminary technical overview of our understanding of the integrated system performance based tests and analyses conducted during the on-sky commissioning campaign with LSSTCam.
The objectives are to synthesize what we have learned about the system in a timely way to inform Early Operations optimization, and to inform the Rubin science community on the progress of the LSSTCam on-sky campaign.
The report is organized into sections that describe major activities during the campaign, as well as multiple aspects of the demonstrated system and science performance.
All of the results presented here are to be understood as work in progress using engineering data and the initial versions of the data processing pipelines; the report is a living document that will be
updated as analyses are refined.
}

% Change history defined here.
% Order: oldest first.
% Fields: VERSION, DATE, DESCRIPTION, OWNER NAME.
% See LPM-51 for version number policy.
\setDocChangeRecord{%
  \addtohist{1}{YYYY-MM-DD}{Unreleased.}{Bechtol}
}


\begin{document}

% Create the title page.
\maketitle
% Frequently for a technote we do not want a title page  uncomment this to remove the title page and changelog.
% use \mkshorttitle to remove the extra pages

% ADD CONTENT HERE
% You can also use the \input command to include several content files.

\section{Introduction}
\label{sec:introduction}

The \VeraRubinObservatory on-sky commissioning campaign using the LSST Camera (hereafter LSSTCam) began on 15 April 2025 and ended on 21 September 2025.
This interim report provides a concise summary of our understanding of the integrated system performance based tests and analyses conducted during the LSSTCam on-sky campaign.
We seek to distill, and to communicate in a timely way, what we have learned about the system to support the transition from Rubin Observatory Construction to Operations.
The report is organized into sections that describe major activities during the campaign, as well as multiple aspects of the demonstrated system and science performance.

\begin{warning}[Preliminary Results]
All of the results presented here are to be understood as work in progress using engineering data and the
initial versions of the data processing pipelines.
It is expected at this stage, immediately following the completion of the on-sky commissioning campaign, that several analyses are still in progress, and that some of the discussion will concern open questions, issues, and anomalies that are actively being worked by the team to enhance the system reliability.
Additional documentation will be provided as our understanding of the demonstrated performance of the as-built system progresses.
\end{warning}

\subsection{Charge}

\begin{note}[Charge Development Historical Note]
    The initial version of the charge developed in September 2025 is provided below for reference.
\end{note}

We identify the following high-level goals for this interim report:

\begin{itemize}

    \item \textbf{Document our current understanding of the integrated system performance} to support systems engineering verification activities associated with demonstrating Construction Completeness \citedsp{sitcomtn-005}.

    \item \textbf{Transfer knowledge to support the transition from Construction to Operations} to inform the Early Operations optimization period and to support the Early Science Program \citedsp{RTN-011}.

    \item \textbf{Inform the Rubin Science Community} on the progress of the on-sky commissioning campaign using LSSTCam.

\end{itemize}

Formal acceptance testing with respect to system-level requirement specifications (\citeds{lse-29} and \citeds{lse-30}) will be recorded using the LSST Verification \& Validation (LVV) system.
By design, several of the analyses presented in this report correspond to system-level requirements, and therefore, this report is anticipated to serve as a verification artifact to support several of those systems engineering activities.

\textbf{The groups within the Rubin Observatory project working on each of the activities and performance analyses are charged with contributing to the relevant sections of the report.}
The anticipated level of detail for the sections ranges from a paragraph up to a page or two of text, depending on the current state of understanding, with \textbf{quantitative performance} expressed as summary statistics, tables, and/or figures.
The objective for this document is to \textbf{summarize the state of knowledge of the system}, rather than how we got there or ``lessons learned''.
The sections refer to additional supporting documentation, e.g., analysis notebooks, other technotes with further detail, as needed.
Given the timelines for commissioning various aspects of the system, it is natural that some sections will have more detail than others.

The anticipated milestones for developing this interim report are as follows:

\begin{itemize}

    \item 18 Sep 2025: Define charge

    \item 22 Sep 2025: On-sky commissioning campaing with LSSTCam completed; start of final construction downtime and its first operations engineering downtime

    \item 8 Oct 2025: Detailed outlines with initial versions of essential figures and performance statistics for report sections made available for internal review (content can be on unmerged development branches); a goal is to help systems engineering with mapping of report content to requirements verification

    \item 15 Oct 2025: Revised drafts of report sections made available for internal review; development branches merged to main branch; editing for consistency and coherency throughout the report

    \item 22 Oct 2025: Start of Construction to Operations Transition Workshop; advanced draft ready for review by full Rubin Observatory team

    \item 31 Oct 2025: Initial version of report is released

\end{itemize}

\section{Executive Summary}
\label{sec:summary}

Executive summary here.

\begin{note}[Versioning Note]
    This interim report provides a preliminary technical overview of the LSSTCam on-sky campaign based on analyses through October 2025.
\end{note}

\subsection{Accomplishments}

\begin{itemize}
    \item \textbf{Accomplishment.} Description.
    \item \textbf{Accomplishment.} Description.
    \item \textbf{Accomplishment.} Description.
\end{itemize}

\subsection{Areas of Ongoing Investigation and Further Development}

\begin{itemize}
    \item \textbf{Issue.} Description.
    \item \textbf{Issue.} Description.
    \item \textbf{Issue.} Description.
\end{itemize}

\section{Sensor Performance and Instrument Signature Removal}
\label{sec:camera}

\begin{itemize}
    \item Usable pixels, effective field of view, fill factor
    \item Read noise
    \item Crosstalk
    \item Dynamic range (brightest and faintest objects)
    \item Summary table of key Camera performance metrics
    \item Section needs at least one figure to visualize focal plane; maybe a flat?
    \item Standard visit (i.e., snaps) evaluation?
\end{itemize}

\subsection{Useable Pixels}

Field of View Area Factor (fA) for System Performance diagram

Could include a figure to show focal plane

\subsection{Sensor Characterization}

Summary discussion on

\subsection{Bright Stars and Moon}

On-sky validation that the Camera can survey the night sky in the presence of bright sources

\subsection{Sensor Anomalies}

Any sensor anomalies that are worth noting

\subsection{LSSTCam Performance during the LSSTCam On-sky Campaign}

Brief summary of LSSTCam performance during the campaign, any open questions, outstanding issues

This could include any discussion on Camera susbsystems, e.g., focal plane optimization, filter exchange system, cryo, camera shutter, that should be highlighted

\section{System Optical Throughput for Focused Light}
\label{sec:throughput}

\begin{itemize}
    \item Standard bandpass; includes the sensors, filters, lenses, mirrors, and (a standard) atmosphere. Measured with CBP? Monochromatic flats with flat field screen?
    \item Imaging depth in multiple bands (LSR-REQ-0090); also express as zeropoint to separate out the effects of image quality; could be comparison to refcats and/or spectrophotometric standards
    \item Figure to show throughput variation of throughput across field of view (LSR-REQ-0109); potentially separating out vignetting and CCD response
    \item Discussion on Sensitivity Factor (fS) in the System Performance Diagram
\end{itemize}

\subsection{Standard Bandpass}

\subsection{Measured Zeropoints}


\section{Measured Sky Backgrounds}
\label{sec:sky_background}

Section Editor: (?)

\begin{itemize}
    \item Comparison to predicted sky background levels
    \item Distribution of limiting surface brightness
\end{itemize}

\section{Delivered Image Quality}
\label{sec:image_quality}

\subsection{Delivered Image Quality Distribution}

\begin{itemize}
    \item Examples of some of our best images to demonstrate system capability
    \item Figure: distribution of PSF FWHM for an ensemble of visits
    \item Figure: PSF size and ellipticity distribution across field of view for an ensemble of visits
\end{itemize}

\subsection{Image Quality Budget}

Discussion on current assessment of the various contributions to the delivered image quality, accomplishments, open questions, outstanding issues

Discussion on current state of characterizing the atmosphere contribution

\subsection{PSF Characterization}

\begin{itemize}
    \item Figure: PSF residuals across the field of view
    \item Figure: measured PSF size as a function of magnitude (could include brighter-fatter discussion)
    \item Figure: measured PSF size as a function of stellar color
    \item Figure: wings of PSF and encircled energy as a function of radius
\end{itemize}

\section{Stray and Scattered Light}
\label{sec:stray_light}

\begin{itemize}
    \item Figure: examples of ghosts; are the features understood?
    \item Figure: examples of stray and scattered light; are the sources of stray and scattered light understood?
    \item Summative discussion on the impacts of stray and scattered light; how prevalent, amplitude and structure of the features, to what extent will additional baffling (e.g., with LWS) mitigate the features
\end{itemize}

\section{System Timing and Dynamics}
\label{sec:system_timing}

\begin{itemize}
    \item Standard Visit Duration (OSS-REQ-0288)
    \item Readout time – discussed with the Camera?
    \item Time Interval Between Visits (OSS-REQ-0289)
    \item Maximum time for operational filter change (OSS-REQ-0293)
    \item Telescope Azimuth Slewing Rate (TLS-REQ-0029)
    \item Telescope Elevation Slewing Rate (TLS-REQ-0159)
    \item Summative assessment on rate of acquiring observations
\end{itemize}

\subsection{Standard Visit Definition}

Discussion on decision to use 30-second exposures

\subsection{Visit Timing and Interval between Visits}

Camera readout time, filter change times

Telescope motion settings, slew and setttle, distribution of time between visits

\subsection{Effective Survey Speed}

Observing efficiency factor (fO) for System Performance diagram

Survey simulations combined with telescope motion capabilities; compare with actual rate of acquiring visits during SV surveys


\section{Data Management}
\label{sec:data_management}

The primary purpose of this section is to describe that data management has been able to support the operational aspects of running Rubin Observatory during commissioning

\begin{itemize}
    \item Calibration products and ISR during commissioning
    \item Brief description (paragraph or two; maybe a table) of data processing campaigns during on-sky commissioning, mainly reporting on the functional capabilities; algorithms and data products are discussed elsewhere; pointers to other references
    \item Figure with representative pixel-level color coadd images?
\end{itemize}



\section{Calibration}
\label{sec:calibration}

\subsection{Astrometry}
\label{sec:astrometry}

\begin{itemize}
    \item Figure: histogram of astrometric repeatability for ensemble of visits
    \item Figure: static camera astrometric distortion model
    \item Figure: average astrometric residuals in focal plane coordinates for ensemble of visits; two-panel figure to show full focal plane and an individual detector
    \item Figure: average E/B mode across ensemble of visits; expect to see mostly a pure E-mode astrometric field indicative of residuals dominated by astrometric turbulence
    \item Item any further discussion of astrometric residuals worth exploring further
\end{itemize}

\subsection{Photometry}
\label{sec:photometry}

\begin{itemize}
    \item Figure: histogram of photometric repeatability for ensemble of visits; panel for each band?
    \item Figure: illumination correction
    \item Figure: average photometric residuals in focal plane coordinates for ensemble of visits; two-panel figure to show full focal plane and an individual detector
\end{itemize}

\appendix

\section{Acknowledgements}

This material is based upon work supported in part by the National Science Foundation through Cooperative Agreements AST-1258333 and AST-2241526 and Cooperative Support Agreements AST-1202910 and AST-2211468 managed by the Association of Universities for Research in Astronomy (AURA), and the Department of Energy under Contract No.\ DE-AC02-76SF00515 with the SLAC National Accelerator Laboratory managed by Stanford University.
Additional Rubin Observatory funding comes from private donations, grants to universities, and in-kind support from LSST-DA Institutional Members.

% Include all the relevant bib files.
% https://lsst-texmf.lsst.io/lsstdoc.html#bibliographies
\section{References} \label{sec:bib}
\renewcommand{\refname}{} % Suppress default Bibliography section
\bibliography{local,lsst,lsst-dm,refs_ads,refs,books}

% Make sure lsst-texmf/bin/generateAcronyms.py is in your path
\section{Acronyms} \label{sec:acronyms}
\input{acronyms.tex}
% If you want glossary uncomment below -- comment out the two lines above
%\printglossaries





\end{document}
