\section{Calibration Systems}
\label{sec:calibration_systems}

We performed comprehensive in-dome measurements with a suite of calibration systems to generate calibration products for LSSTCam.
Calibration data not requiring external illumination included biases and darks.
We operated the Flat Field Projector with single-LED flats and generated Photon Transfer Curves (PTCs).
We exercised both calibration system sources: a white-light LED with a fiber spectrograph and photodiode plus electrometer, and a tunable laser with the same metrology chain.
We used the Collimated Beam Projector (CBP) for filter scans (and no-filter scans) and to acquire crosstalk spot data.
These activities validated the calibration pathways and established in-situ sensor performance baselines.

\begin{figure}[h]
    \centering
    \includegraphics[width=\textwidth]{figures/calibration_systems.png}
    \caption{Our flat field screen illuminated by our single LEDs and pairs of single LEDs. The LEDs from left to right are named Royal Blue, both Royal Blue and Green, Green, Lime, both Lime and Deep Red, Deep Red)}
    \label{fig:calib_systems}
\end{figure}

\begin{itemize}
    \item Calibration Data without external illumination
    \begin{itemize}
        \item Biases
        \item Darks
    \end{itemize}
    \item Calibration System Sources
    \begin{itemize}
        \item White Light (LED) w/ Fiber Spectrograph and Photodiode + Electrometer
        \item Tunable Laser w/ Fiber Spectrograph and Photodiode + Electrometer
    \end{itemize}
    \item Flat Field Projector system
    \begin{itemize}
        \item (Single) LED flats
        \item Photon Transfer Curve
    \end{itemize}
    \item Collimated Beam Projector (CBP)
    \begin{itemize}
        \item Filter scans (and no-filter scans)
        \item Crosstalk spots
    \end{itemize}
\end{itemize}

\subsubsection{Summary}

All in-dome calibration systems were exercised end-to-end, and their associated calibration products validated through the de-trending of on-sky observations. :contentReference[oaicite:3]{index=3}