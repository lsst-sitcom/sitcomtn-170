\section{Introduction}
\label{sec:introduction}

The \VeraRubinObservatory on-sky commissioning campaign using the LSST Camera (hereafter LSSTCam) began on 15 April 2025 and ended on 21 September 2025.
This interim report provides a concise summary of our understanding of the integrated system performance based tests and analyses conducted during the LSSTCam on-sky campaign.
We seek to distill, and to communicate in a timely way, what we have learned about the system to support the transition from Rubin Observatory Construction to Operations.
The report is organized into sections that describe major activities during the campaign, as well as multiple aspects of the demonstrated system and science performance.

\begin{warning}[Preliminary Results]
All of the results presented here are to be understood as work in progress using engineering data and the
initial versions of the data processing pipelines.
It is expected at this stage, immediately following the completion of the on-sky commissioning campaign, that several analyses are still in progress, and that some of the discussion will concern open questions, issues, and anomalies that are actively being worked by the team to enhance the system reliability.
Additional documentation will be provided as our understanding of the demonstrated performance of the as-built system progresses.
\end{warning}

\subsection{Charge}

\begin{note}[Charge Development Historical Note]
    The initial version of the charge developed in September 2025 is provided below for reference.
\end{note}

We identify the following high-level goals for this interim report:

\begin{itemize}

    \item \textbf{Document our current understanding of the integrated system performance} to support systems engineering verification activities associated with demonstrating Construction Completeness \citedsp{sitcomtn-005}.

    \item \textbf{Transfer knowledge to support the transition from Construction to Operations} to inform the Early Operations optimization period and to support the Early Science Program \citedsp{RTN-011}.

    \item \textbf{Inform the Rubin Science Community} on the progress of the on-sky commissioning campaign using LSSTCam.

\end{itemize}

Formal acceptance testing with respect to system-level requirement specifications (\citeds{lse-29} and \citeds{lse-30}) will be recorded using the LSST Verification \& Validation (LVV) system.
By design, several of the analyses presented in this report correspond to system-level requirements, and therefore, this report is anticipated to serve as a verification artifact to support several of those systems engineering activities.

\textbf{The groups within the Rubin Observatory project working on each of the activities and performance analyses are charged with contributing to the relevant sections of the report.}
The anticipated level of detail for the sections ranges from a paragraph up to a page or two of text, depending on the current state of understanding, with \textbf{quantitative performance} expressed as summary statistics, tables, and/or figures.
The objective for this document is to \textbf{summarize the state of knowledge of the system}, rather than how we got there or ``lessons learned''.
The sections refer to additional supporting documentation, e.g., analysis notebooks, other technotes with further detail, as needed.
Given the timelines for commissioning various aspects of the system, it is natural that some sections will have more detail than others.

The anticipated milestones for developing this interim report are as follows:

\begin{itemize}

    \item 18 Sep 2025: Define charge

    \item 22 Sep 2025: On-sky commissioning campaing with LSSTCam completed; start of final construction downtime and its first operations engineering downtime

    \item 8 Oct 2025: Detailed outlines with initial versions of essential figures and performance statistics for report sections made available for internal review (content can be on unmerged development branches); a goal is to help systems engineering with mapping of report content to requirements verification

    \item 15 Oct 2025: Revised drafts of report sections made available for internal review; development branches merged to main branch; editing for consistency and coherency throughout the report

    \item 22 Oct 2025: Start of Construction to Operations Transition Workshop; advanced draft ready for review by full Rubin Observatory team

    \item 31 Oct 2025: Initial version of report is released

\end{itemize}