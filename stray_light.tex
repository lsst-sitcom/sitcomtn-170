\section{Stray and Scattered Light}
\label{sec:stray_light}

The unique, wide-field design of the Simonyi Survey Telescope it particularly susceptible to stray and scattered light. Since LSST seeks to investigate the low-surface-brightness universe \citep{2019ApJ...873..111I}, it is important to identify, model, and mitigate stray-light artifacts. This becomes particularly important in the context of the  the delayed installation of the light--wind screen \citep[LWS;][]{Marchiori:2024}. A general overview of the investigation of stray light features during commissioning can be found in \citet{SITCOMTN-160}, while a detailed investigation of the most prominent stray light feature (the ``scratched tape'') is discussed in \citet{SITCOMTN-166}. Here, we provide a brief summary of those efforts and the impact of stray and scattered light.


\begin{itemize}
  %ADW: Ghosts should be discussed elsewhere
  %\item Figure: examples of ghosts; are the features understood?
  \item Figure: examples of stray and scattered light; are the sources of stray and scattered light understood?
  \item Summative discussion on the impacts of stray and scattered light; how prevalent, amplitude and structure of the features, to what extent will additional baffling (e.g., with LWS) mitigate the features
\end{itemize}
