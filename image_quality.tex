\section{Delivered Image Quality}
\label{sec:image_quality}

\subsection{Delivered Image Quality Distribution}

\begin{itemize}
    \item Examples of some of our best images to demonstrate system capability
    \item Figure: distribution of PSF FWHM for an ensemble of visits
    \item Figure: PSF size and ellipticity distribution across field of view for an ensemble of visits
\end{itemize}

\subsection{Image Quality Budget}

Discussion on current assessment of the various contributions to the delivered image quality, accomplishments, open questions, outstanding issues

Discussion on current state of characterizing the atmosphere contribution

\subsection{PSF Characterization}

The characterization of the point spread function (PSF) provides a key measure of the image quality and calibration performance achieved during LSSTCam commissioning.

\subsubsection{Focal Plane Residuals}

Average PSF size residuals across the focal plane are at or below the 0.5\% level, indicating sub-percent accuracy in the PSF modeling (Fig.~\ref{fig:psf_residuals}).
When the residuals are stacked across ensemble visits, distinct spatial structures become visible.
A ring-like pattern at the edge of the field corresponds to vignetting effects.
Small amplifier-level offsets are observed on e2v sensors.
Most notably, a “blob” pattern appears on ITL sensors, consisting of circular features within each device.

These ITL features are highly correlated with the height map of each sensor measured during laboratory testing at SLAC.  (Fig.~\ref{fig:psf_height}).
Blink comparisons between the PSF residual maps and the measured height maps show an excellent correspondence.
This confirms that local defocus from sensor height variations produces the observed PSF structure.
The PSF thus serves as a secondary map of focal-plane topography, which can be incorporated into future models to further reduce residuals.

\begin{figure}[h]
    \centering
    \includegraphics[width=\textwidth]{figures/focalplane_T_residuals.png}
    \caption{Average PSF size residuals across the LSSTCam focal plane.
    Residuals are at the $\leq$0.5\% level.
    The ring at the edge corresponds to vignetting, small offsets appear on e2v sensors, and the blob pattern on ITL sensors matches the laboratory-measured sensor height map from SLAC.}
    \label{fig:psf_residuals}
\end{figure}

\begin{figure}[h]
    \centering
    \includegraphics[width=0.45\textwidth]{figures/detector_height.png}
    \includegraphics[width=0.495\textwidth]{figures/T_residuals.png}
    \caption{Size residuals are highly correlated with the  laboratory-measured sensor height map from SLAC.}
    \label{fig:psf_height}
\end{figure}

\subsubsection{Brighter–Fatter Correction}

The dependence of PSF size on source flux—the brighter–fatter effect—has been modeled and corrected using an electrostatic approach.
This new model achieves accuracy better than 0.5\%, satisfying and exceeding both Rubin internal requirements and external dark-energy science goals.
Comparisons of residual PSF size as a function of flux show that the electrostatic model removes the previous over-correction observed in the default pipeline.
Implementation of this improved correction in standard processing is in progress.

\begin{figure}[h]
    \centering
    \includegraphics[width=0.6\textwidth]{figures/brighterFatter.png}
    \caption{Brighter–fatter correction performance.
    The blue curve shows the uncorrected flux-dependent PSF size residuals, the red curve the current pipeline correction, and the black curve the new electrostatic model.
    The latter achieves sub-0.5\% accuracy and removes the over-correction seen previously.}
    \label{fig:brighter_fatter}
\end{figure}

\subsubsection{Ongoing and Future Work}

Several enhancements to PSF modeling are under active development.
A chromatic PSF model has been demonstrated and implemented in the pipeline but is not yet enabled by default.
Work is underway to transition from per-CCD to full focal plane PSF modeling.
Efforts are also in progress to shift from pixel coordinates to sky coordinates for PSF interpolation and to incorporate laboratory-measured sensor height information directly into the model.
Each of these improvements is expected to further reduce PSF residuals and improve astrometric and shear accuracy.

PSF characterization during LSSTCam commissioning demonstrates that the system achieves sub-percent modeling accuracy across the focal plane.
Residual patterns correlate strongly with known sensor characteristics, validating both the physical understanding and calibration fidelity of the system.
With the inclusion of chromatic and full-focal-plane models in upcoming releases, PSF performance is expected to reach the design specifications required for precision cosmology.
   