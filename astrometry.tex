\subsection{Astrometry}
\label{sec:astrometry}



Astrometric calibration establishes the geometric fidelity of the LSSTCam imaging system and ensures that object positions are consistent across visits and with external reference catalogs.
The analysis presented here summarizes the current performance achieved during the Science Validation campaign and identifies the dominant sources of residual error.

\subsubsection{Global Calibration}

The global astrometric calibration is performed using the \texttt{gbdesAstrometricFit} model, which fits for source proper motions and parallaxes across all visits.
Differential chromatic refraction (DCR) fitting will be enabled in upcoming processing runs.
Recent improvements include the introduction of multiprocessing and the switch from small tract-level runs to larger HEALPix regions.
These changes significantly reduced total processing time and improved overall convergence stability.

The resulting astrometric repeatability meets design expectations in most regions of the sky observed with full focal-plane dithers.
Figure~\ref{fig:am1_map} shows the AM1 metric, defined as the standard deviation of the separation between pairs of objects approximately five arcminutes apart.
Typical values are at the $\sim$10~milliarcsecond (mas) level, with localized regions showing higher scatter that are under investigation.

\begin{figure}[h]
    \centering
    \includegraphics[width=\textwidth]{figures/am1_map.png}
      \includegraphics[width=0.7\textwidth]{figures/astrometric_repeat_hist.png}

    \caption{Map of the AM1 astrometric repeatability metric across the Science Validation survey region.
    The majority of the field achieves 10~mas RMS repeatability, consistent with design specifications.}
    \label{fig:am1_map}
\end{figure}

\subsubsection{Residual Patterns and Atmospheric Contribution}

Mean astrometric residuals binned by position in the focal plane are shown in Figure~\ref{fig:astrometry_residuals}.
The residuals in both X and Y directions are reveal effects at the chip level.
For example the features on the ITL detectors are consistent with the laboratory measured sensor height maps.
These higher order distortions remain at the individual chip level and will be modeled out in future iterations.

\begin{figure}[h]
    \centering
    \includegraphics[width=\textwidth]{figures/astrometry_residuals.png}
    \caption{Mean astrometric residuals across the focal plane in X (left) and Y (right) directions.
    The spatial structure is consistent with atmospheric turbulence and small chip-level distortions.}
    \label{fig:astrometry_residuals}
\end{figure}

Because LSSTCam uses relatively short 30-second exposures, atmospheric turbulence introduces correlated position shifts across the field.
A Gaussian-process model has been developed to represent this atmospheric component and is being tested for inclusion in the standard pipeline.
This model reproduces the observed E-mode correlation structure seen in the residuals and reduces the single-visit scatter by roughly 50\% when applied.

\begin{figure}[h]
    \centering
    \includegraphics[width=\textwidth]{figures/per_visit_astrometry_residuals.png}
     \includegraphics[width=\textwidth]{figures/gaussian_process_model.png}
    \caption{Gaussian-process modeling of atmospheric turbulence.
    The model captures large-scale correlated patterns in the residuals and reduces single-visit astrometric scatter by about 50\%.}
    \label{fig:gaussian_process_model}
\end{figure}

After subtraction of the modeled atmospheric term, stacked residual maps reveal additional fine-scale structure, including chip-dependent distortions and faint tree-ring patterns on ITL sensors.
These effects are below the atmospheric level but are now detectable due to reduced noise and will be incorporated into refined models.

\subsubsection{Refined Camera Distortion Model}

Astrometric solutions have been used to derive an updated camera distortion model, replacing the pre-commissioning geometric model from \texttt{obs\_lsst}.
This new model reflects the as-built optical system and the measured alignment of detectors in situ.
The difference between the two models is shown in Figure~\ref{fig:camera_model_diff}.
After removing a first-order affine transformation, small chip-to-chip residuals and systematic trends between ITL and e2v sensors become visible.

\begin{figure}[h]
    \centering
    \includegraphics[width=\textwidth]{figures/camera_model_diff.png}
    \caption{Comparison between the pre-commissioning camera model and the astrometry-derived distortion model.
    Residuals after removing a first-order affine term reveal fine-scale chip-to-chip differences and sensor-dependent trends.}
    \label{fig:camera_model_diff}
\end{figure}

The refined camera model is now incorporated into single-frame processing, improving the accuracy of instrumental calibration and the consistency of WCS solutions.

\subsubsection{Summary}

Astrometric repeatability with LSSTCam currently achieves approximately 10~mas precision, consistent with design specifications in regions with focal-plane-scale dithers.
Residuals at the single-visit level are dominated by atmospheric turbulence and are well described by a Gaussian-process model that reduces scatter by about a factor of two.
Fine-scale detector distortions are now measurable and are being folded into updated calibration models.
A refined, on-sky camera distortion model has been adopted for single-frame processing and continues to improve the overall geometric calibration of the system.
Future work will integrate DCR fitting, parallaxes, and the Gaussian-process atmospheric model directly into the global solution for DR1.
